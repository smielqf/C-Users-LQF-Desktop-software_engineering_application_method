\documentclass[11pt, a4paper]{article}
\usepackage{xeCJK}
\begin{document}
    \title{Homework 1: A Small Program Fragment}
    \author{林奇峰\\
     Qifeng Lin\\
     Student ID:17214656 }
    \date{\today}
    \maketitle

\noindent Description:

    {\sffamily\bfseries process} Inc = {\sffamily\bfseries while} {\slshape true} {\sffamily\bfseries do} if {\slshape x} < 200 {\sffamily\bfseries then} {\slshape x}:={\slshape x}+1 {\sffamily\bfseries od}

    {\sffamily\bfseries process} Dec = {\sffamily\bfseries while} {\slshape true} {\sffamily\bfseries do} if {\slshape x} > 0 {\sffamily\bfseries then} {\slshape x}:={\slshape x}-1 {\sffamily\bfseries od}

    {\sffamily\bfseries process} Reset = {\sffamily\bfseries while} {\slshape true} {\sffamily\bfseries do} if {\slshape x} = 200 {\sffamily\bfseries then} {\slshape x}:=0 {\sffamily\bfseries od}

    \begin{flushright}
      {\slshape is x always between (and including) 0 and 200?}
    \end{flushright}

\noindent Solution:
    {\slshape x} is not always between (and including) 0 and 200.

    For example, with the fact that the initial value of {\slshape x} is 200, the {\sffamily\bfseries process Dec}  and the {\sffamily\bfseries process Reset}  run concurrently. Then the execution order may be as follow showing:

    if {\slshape x} = 200 {\sffamily\bfseries ->} if {\slshape x} > 0 {\sffamily\bfseries ->} {\slshape x}:=0  {\sffamily\bfseries ->} {\slshape x}:={\slshape x}-1

    Therefore, the final value of {\slshape x} is -1 and it is not between (and including) 0 and 200.
    
    To guarantee the value, it may use some techniques of process synchronization like lock.

\end{document} 